\chapter{Dataset}

Why do we use a dataset?
- learning
- some research make them freely available to test

\textbf{UEC FOOD-100} and \textbf{UEC FOOD-256} are datasets used for food localization and recognition.

The UEC FOOD-100 dataset can be found in \footnote{Dataset can be found at \url{http://foodcam.mobi/dataset100.html}}. It was created in 2012 and presented in \cite{Matsuda2012a}.

It contains 100 types of food, mainly Japanese food. Each kind is represented by at least 100 samples.

\begin{figure}[h]
    \includegraphics[width=8cm, height=8cm]{img/multiple_food_items_1}
    \includegraphics[width=8cm, height=8cm]{img/multiple_food_items_2}
    \includegraphics[width=8cm, height=8cm]{img/multiple_food_items_3}
    \includegraphics[width=8cm, height=8cm]{img/multiple_food_items_4}
    \caption{Example of pictures with multiple food items from UEC FOOD 256}
    \label{fig:presentation_multiple_food_items}
\end{figure}

As presented in figure \ref{fig:presentation_multiple_food_items}, a photo can contain more than one food items. The dataset contains files to indicate bounding boxes marking the location of a food items.

UEC FOOD-256 can be found in \footnote{Dataset can be found at \url{http://foodcam.mobi/dataset256.html}}. It was presented in \cite{Kawano2015} in 2015. It contains  the 100 types of food from UEC FOOD-100 plus 156 new ones. The newly introduced food kinds are more international dishes. As for FOOD 100, every food photo has a bounding box indicating the location of the food item. The categories with the most number of picture is rice (620) and miso soup (728).