\documentclass[aspectratio=169]{beamer}

\usepackage[utf8]{inputenc}
\usepackage[english]{babel}

\let\oldsection\section
\renewcommand{\section}[1]{
    \oldsection{#1}	
    \subsection{}
}

\newenvironment{myframe}[1][t]{\begin{frame}[#1]{\secname}{\subsecname}}{\end{frame}}

\usetheme{cranfielduniversity}

\author{Baptiste NOGARET}
\title{Food log analysis}

\supervisor{Dr Stefan RÜGER}

\setcounter{tocdepth}{1}

\begin{document}
	
	\begin{frame}[plain]
		\titlepage
	\end{frame}
    
    \begin{frame}
        \frametitle{Table of Contents}
        \tableofcontents
    \end{frame}
    
    \section{Introduction}
    
    \subsection{Why a food log analysis system?}
    
	\begin{myframe}
        Obesity / overweight in the world (lead to obesity, high blood pressure ... + cost)
        Best way to fight it: watch over what we eat
        With the progress of machine learning + access to more and more powerful device + widespread smartphone
        Proposition to automatize it
	\end{myframe}
    
    \subsection{Challenge of food recognition}
    
    \begin{myframe}
        High intra-class variability
        Environment
        Way of taking the picture (zoom in or out)
        
        (Example)
    \end{myframe}
    
    \subsection{Example of Implementation: Food Log}
    
    \begin{myframe}
        
    \end{myframe}
    
    \subsection{Overall process}
    
    \begin{myframe}
        Have a dataset of picture (from user ...)
        extract characteristic
        Segmentate
        Classify (learn)
        Estimate calory / quality intake
        Focus on the first three parts
    \end{myframe}
    
    \section{Dataset}
    
     \begin{myframe}
         Available dataset
         PFID
         UECFOOD 100 and 256
         
         Choose UECFOOD 256: segmentation, 256 classes (challenging)
         Different international categories ex: Japanese (rice, )
    \end{myframe}
    
    \begin{myframe}
        Picture to illustrate the multi items
    \end{myframe}
    
    \section{Feature description}
    
    \subsection{Bag of visual words}
    
    \begin{myframe}
        Describe briefly SIFT / SURF
        Overall process
        \begin{enumerate}
            \item Keypoints detection
            \item Keypoints description
            \item Clustering to get N words
            \item Express each picture by an histogram of visual words
        \end{enumerate}
    \end{myframe}
    
    \subsection{Local binary pattern}
    
    \begin{myframe}
        content...
    \end{myframe}
    
    \subsection{Color moments and histograms}
    
    \begin{myframe}
        Mean, variance
        Hu moment?
        
        Color histogram (joint histogram)
        HSV channel
    \end{myframe}
    
    \section{Classifiers}
    
    \subsection{Tree and random forest}
    
    \begin{myframe}
        content...
    \end{myframe}
    
    \subsection{SVM}
    
    \begin{myframe}
        content...
    \end{myframe}
    
    \subsection{CNN}
    
    \begin{myframe}
        Can be used for more than class
    \end{myframe}
    
    \section{Segmentation}
    
    \begin{myframe}
        content...
    \end{myframe}
    
    \section{Result}
    
    \subsection{Segmentation}
    
    \begin{myframe}
        content...
    \end{myframe}
    
    \subsection{Classification}
    
    \begin{myframe}
        content...
    \end{myframe}
    
    \subsection{Segmentation followed by classification}
    
    \begin{myframe}
        COmpare to other results
    \end{myframe}
    
    \section{Future work}
    
    \begin{myframe}
        COmpare to other results
    \end{myframe}
    
    \section{Questions?}
    
    \begin{myframe}
        Thank you for your attention.
        Do you have any questions?
    \end{myframe}


%    \begin{myframe}
%        Dataset
%        
%        Picture with multiple item
%
%         \begin{columns}
%             
%             \column{0.5\textwidth}
%             This is a text in first column.
%             $$E=mc^2$$
%             \begin{itemize}
%                 \item First item
%                 \item Second item
%                \end{itemize}
%                
%                \column{0.5\textwidth}
%                This text will be in the second column
%                and on a second tought this is a nice looking
%                layout in some cases.
%            \end{columns}
%    \end{myframe}
   
\end{document}