\documentclass[12pt]{book}

\usepackage[T1]{fontenc}
\usepackage[utf8]{inputenc}
\usepackage[english]{babel}
\usepackage{graphicx}
\usepackage{tabulary}
\usepackage{mathtools}
\usepackage{amsfonts}
\usepackage[autostyle]{csquotes}

\usepackage{textcomp}                   % For the special character ° and TM

\usepackage{cranfieldthesis}            % Use the custom "cranfieldthesis" LaTeX style file. 

\usepackage[sorting=none, backend=biber, style=numeric]{biblatex}
\usepackage[colorlinks=false]{hyperref}

% By default, LaTeX uses a serif font - these are traditionally thought to be
% easier to read.   If you'd prefer sans-serif, please uncomment the 
% following line.
%\renewcommand{\familydefault}{\sfdefault}

\newcommand{\TM}{\textsuperscript{TM}~}
% \newcommand{\source}[1]{\caption*{\hfill Source: {#1}} }

% Example parameters for a typical taught MSc course
\title{Automated food log analysis}
\author{NOGARET Baptiste}
\date{\today}
\school{Aerospace, Transport and Manufacturing}
\course{Computational and Software Techniques in Engineering}
\degree{Master of Science}
\academicyear{2015--2016}
\supervisor{Dr RÜGER Stefan}
\copyrightyear{2016}

\addbibresource{library.bib}
\addbibresource{my_references.bib}

\newcommand{\E}{\mathbb{E}}
\newcommand{\N}{\mathbb{N}}
\newcommand{\Var}{\operatorname{Var}}
\newcommand{\argmin}{\operatorname {arg\,min} }
\newcommand{\argmax}{\operatorname {arg\,max} }
\newcommand{\sign}{\operatorname {sign} }

\begin{document}


% Front matter
\frontmatter

% Standard-Form Title Pages
\maketitle

% Declaration of authorship
\include{declaration_authorship}

% Abstract and Keywords
\begin{abstract}
    
    Automated food log is a promising exemplar of image analysis which allows users to upload their meal pictures that are processed to keep track automatically of one's food intake. It is a challenging problem due to the high variability of dishes (picture conditions, various types, plating).
    
    With this purpose, the presented master thesis describe a process for simultaneous localisation and recognition. To tackle this problem, several feature descriptors and classifiers were sought to obtain the highest efficiency. From the experiments, the leading method is based on two steps. First, a convolutional neural network pre-trained to detect salient objects is applied on each image to generate bounding boxes for every food area and second, an additional convolutional neural network is used in combination of random forest to recognize the food in each bounding box.
    
    Evaluated on the UEC-FOOD 256 dataset, the method enhances the current best segmentation algorithm with 74\% of top-1 accuracy. Overall, an accuracy of 28 \% is obtained.
    
    \section*{Keywords}
    Food log; Photo; Localisation; Classification; Convolutional neural network, Random Forest
\end{abstract}

% Table of Contents
\sstableofcontents

% List of Figures
\sslistoffigures

% List of Tables
\sslistoftables

% The list of abbreviations can't be automatically generated so you need to populate it yourself
\begin{listofabbreviations}
    \abbrev{CNN}{Convolutional Neural Network}
    \abbrev{LBP}{Local Binary Pattern}
    \abbrev{SIFT}{Scale-Invariant Feature Transform}
    \abbrev{SURF}{Speeded Up Robust Features}
    \abbrev{SVM}{Support Vector Machine}
\end{listofabbreviations}

% Acknowledgements
\chapter{Acknowledgements}

I am really grateful to Dr. Stefan Rüger, my supervisor for the project, to have proposed this subject. His guidance and valuable advice were particularly helpful to realise the thesis.

Moreover, I would like to thank the University of Technology of Compiègne for giving me the opportunity to study one year in Cranfield University. I would also like to thank Cranfield University for its facilities.

I would like to express my gratitude to M. Kazu Shimoda and Pr. Keiji Yanai of the University of Tokyo that provided enlightenments and further details on their work and datasets.

%% Main Matter
%
% This is where we include the main thesis content.
%
\mainmatter


\chapter{Introduction}

Risk of obesity \cite{Mokdad2003} (uk, world) "Overweight and obesity were significantly associated with diabetes, high blood pressure, high cholesterol, asthma, arthritis, and poor health status". Obesity is strongly associated with several major health risk factors.

Diabetes:
- fast growing (current ... in to ... in) with forecast for ... to be ...
- lead to high mortality
- treatment cost. \cite{Zhang2010}: in 2010: 12 \% of the total whorlwide health expenditure is spent on diabetes and will continue to increase.

Combination of drugs and food intake control have shown great results.

One of the best way to fight it: watch over what we eat. Associated lifestyle changes and lose weight. Use as a prevention tool for population at risk

studies such as \cite{Burke2011a} show the benefit of reporting its daily diet to lose weight and improve the quality of its foos intake

Also a way to ... eat disorders

Currently, manually ... self reporting, using paper diaries : hard to do + cost a lot + have some problem (people tend to underestimate) + need a trained patient
tedious, prone to error as the useer tend to under-estimate its intake

At the same time, improvement of the classification methods: example on Image Net results \cite{Russakovsky2015}:
1000 classes, more than 1,2 million images
Every year since 2010
Numerous institutions (university, tech companies) are participated
As described in figure ..., the mean error for each class for classification and localization has been greatly reduced between 2010 and 2014

Recently: proposition automize it. With the widespread use of smartphone, people can easily take pictures of a good quality. People are already taking picture of their food and posting them on website such as Food Gawker, Instagaram, Flickr, Yelp or 

That's why, over the past few years, people ... automated it. Assist patient and their medical personnel
Extends the reach of care in a cost effective ways and counters some of the previous problem (still pb with the elder / people who don't have access to smartphone)

Part of the rise of e-healthcare / m-healtchare \cite{Hillestad2005, Menachemi2011}

food recognition: promising applications of image processing and machine learning. Estimate food intake and people's habit

Overall process:
extract characteristic (possible features are invariant of the liminosity, orientation, scale, ...)segmentate, classify, get calori value or a simplified version (using for example the ... systems), keep log and beng able to visualize it over the year

Feature description: key to achieve good object detection and image categorization

In this thesis: focus on the first two phases

Already have numerous challenges:
large number of food items
variation in appearance and shape
different way to server it
environmental condition
--> lead to a high inter-class variability
challenging task for the human


% show pictures as examples


\chapter{Previous work} \label{sec:previous_work}

A profusion of techniques 

\section{Food localisation}

%% Color and edge segmentation
%% 11.1

A way to localize food is based on edge detection and colour segmentation.

In \cite{Thendral2014a}, Thendral et al. describes and compare these two methods to localise an orange in a picture. It applied these methods on a small dataset of 20 orange images (only one orange per image), with different lighting conditions and backgrounds (pictures are taken from the Internet).
In more details, the edge-based segmentation apply the canny-edge segmentation, then apply non-maximum suppression to eliminate noises. Then, each pixels are classified.
The colour-based segmentation normalised the lightning condition with a Gaussian low-pass filter, convert the RGB image into a $L * a * b$
\begin{enumerate}
    \item Gaussian low pass filter to normalize the lightning condition
    \item convert the image from RGB representation to $L * a * b$
    \item use the $a$ channel to classify each pixel as \enquote{fruit} or \enquote{non-fruit}
    \item remove small object
    \item fill the binary image regions and holes
\end{enumerate}
For orange detection, the colour segmentation has an higher accuracy. Yet, it is very hard to generalise this method.

%% Circle
%% 2.2 (partial) + 6.3 + \cite{Dehais2015} ?

An other method for food detection relies on circle detection. Indeed, food items are often served in a round shape container such as a bowl, pan or plate.

In \cite{Wazumi2011}, the authors describe their use of the Hough transformation. The purpose of this technique is to find approximations of instances a certain class of shapes by a voting procedure.
In this paper it is used to detect circles assuming the food is only contained in round plates or bowls. Keeping only the central part of the circle, the segmentation is then fed to the recognition process.

%% CNN
%% DCNN: * ? + 9.1 (in food intake)
%% Presentation of UEC FOOD 100 and 256

A more recent development is the used of convolutional neural network.

In \cite{Shimoda2015}, the authors presents their segmentation process based on a pre-trained deep CNN.

The proposed pipeline is composed of 6 main steps:
\begin{enumerate}
    \item detect all the possible bounding box (maximum 2000 per image) using selective search
    \item cluster the bounding box, using the ration of intersection over union (IOU, also call overlap ratio) to obtain 20 at most.
    \item a Deep CNN for all the selected bounding box to get a saliency map. The DCNN is modelled on AlexNet CNN, was pre-trained on the Salient Object Subitizing (14 000 everyday pictures) dataset and fine-tuned on UEC FOOD 100.
    \item use the GrabCut algorithm to extract the foreground region from the food area. GrabCut is an iterative method using graph cuts to extract foreground from background based on an initial guess.
    \item In case of overlapped bounding box, the authors proposed to apply the non-maximum suppression (NMS) algorithm.
\end{enumerate}

The authors apply this process on the UEC-FOOD 100 dataset and PASCAL VOC 2007. The latter is used for object detection and recognition of 20 common classes (train, tv, cat, human ...)). These two datasets use bounding box to spot items. A segmentation is correct if the overlap ratio exceeds 50\% between the predicted and the ground truth bounding box.

UEC FOOD 100 \footnote{Dataset can be found at \url{http://foodcam.mobi/dataset100.html}} is a dataset created in \cite{Matsuda2012a} containing 100 types of food, mainly Japanese food, and is composed of 9060 pictures. Thus, an image can contain multiple dishes. That's why each food picture has is associated with the bounding box coordinates indicating the food localisation.

For UEC-FOOD 100, the authors obtain 49.9\% mean average accuracy and 58.7\% for PASCAL VOC 2007.

In \cite{Bolanos2016}, the authors use a pre-trained DCNN to classify each pixel as food or non-food. The DCNN is exactly the same structure as \enquote{GoogleNet}, a neural network composed of 22 layers and first used on ILSVRC14 (ImageNet Large Scale Visual Recognition Competition 2014). On UEC-FOOD 256, the authors obtain 60\% of accuracy.

UEC FOOD 256 \footnote{Dataset can be found at \url{http://foodcam.mobi/dataset256.html}} is presented in \cite{Kawano2015} and it is an extension of UEC FOOD 100 (same creators' team). It adds 156 kinds of food from all the over the world (French, Italian, Vietnamese, American, ...). As for FOOD 100, every food photo has a bounding box indicating the food location.

\section{Food recognition}

%Food recognition
%
%Using SVM:
%
%Local using BOW: 3.3
%global feature:
%Colour and texture description: 
%Spatial pyramid:

Over the last few years, authors have focused on food recognition. Various methods were tried. For feature extraction, it often combines colour and texture descriptors, global and local.

In \cite{Chen2009}, the authors provide two simple food classification baseline methods for PFID. PFID (stands for Pittsburgh fast-food image dataset) \footnote{Dataset can be found at \url{http://pfid.rit.albany.edu/}} was presented in \cite{Chen2009} in summer 2008 from the collaboration of Intel Labs Pitssburgh, Columbia and Carnegie Mellon universitie. It is one of the first mature datasets released for food recognition.

It contains 101 meals (categories) from 11 popular fast food chains found in the USA with images and videos captured in both restaurant conditions and controlled lab setting. It contains foods such as chickens, sandwiches, salads, burgers and drinks from Arby's, Bruggers Bagels, Dunkin Donuts, KFC, McDonalds, Panera, Pizza hut, Quiznos, Subway, Taco Bell and Wendy's.

The authors provide :
\begin{itemize}
    \item Colour histogram and SVM classifier. They obtain a mean accuracy of around 12\%.
    \item Bag of SIFT features and SVM classifier. They obtain a mean accuracy of around 25\%.
\end{itemize}

In \cite{Zong2010}, the authors use a local texture feature and their spatial distribution to classify food images from the PFID.

The author use the Bag-Of-Word method; SIFT for detection of the keypoints and LBP for description. The shape context algorithm is used to keep the spatial relationship between codewords (for each image, compute the histogram of one word compared to the others / then mean of the histograms).

For the classification, the authors pick the smallest cost between an image and a food category. For each interest points found with SIFT in the image, we associate a similarity between the point and each visual words of the codebook. The similarity function is based on the Bhattacharyya distance. Then, the shape context between the point of interests and the visual word is calculated and a cost is deduced for each food category. The category with the smallest cost is chosen.

Regrouping the different pictures in 6 main groups the PFI dataset (sandwiches and wraps, meat, salads, donuts, hamburger and miscellaneous), they obtain an average accuracy of 66\%.

Moreover, Fast foods, as they are standardized and have nutrition information available online, can easily be used to measure the calories. In \cite{Wen2009}, the authors are using the PFID's videos to estimate energy intake of a meal.

In \cite{Bosch2011}, the authors use local and global features to identify the food consumed.
For the global features, they use colour properties (entropy, histogram and moments) with texture information provided by Gabor filters. They add local features with the Bag-Of-Features, using SIFT for detection and SIFT, steerable filters and DAISY descriptors. To classify, they use SVM (using the Radial Basis function kernel).
On a in-house dataset (28 classes, 179 images), the authors obtain 86\% of accuracy.

% 4.3 (PFID)

In \cite{Yang2010}, the authors use a novel feature, named PFD (for pairwise local feature distribution) for food recognition and SVM. Then, they apply it on the Pittsburgh fast-food image dataset (PFID) dataset.

The different steps of this method are:
\begin{enumerate}
    \item classify each pixel in one of the categories between beef, chicken, pork, bread, vegetable,  tomato/tomato sauce, cheese/butter, egg/other and background. For classification, they use the Semantic Texton Forest, method based on local characteristics. It was previously trained on 16 manually-labelled pictures.
    
    \item Global ingredient representation (GIR): for the 8 food categories, it sum up the soft label of all the ingredient pixel and normalize by the number.
    
    \item PFD: geometric pairwise feature on N ingredient pixels (picked randomly, thus N / 2 pairs):
    \begin{itemize}
        \item log of the distance
        \item orientation
        \item soft label of the midpoint
        \item soft label of each pixel along the line connecting the pair of pixels
        \item joint feature (a mixed of the above characteristics)
    \end{itemize}
    Accumulate the pairwise values into a distribution (using a multi-dimensional histogram of either 8 or 12 bins), weighted by the soft labels of the two pixels. Each pixel is mapped to its closest bin in the histogram.
    Then, normalization of the histogram.
\end{enumerate}

For the PFID dataset, they obtain an accuracy between 19\% and 28\% for each of the 61 categories.
When they pick the 6 major types of food, they get almost 80\% of accuracy.

% ETHZ Food 101
% 2.1

In \cite{Bossard2014}, the author use the Random forest clustering algorithm to create superpixels (selecting only the discriminative one). On these superpixels, a dense SURF and L*a*b* color value is computed and encoded with improved fisher vectors (IFV) with Gaussian mixture model (GMM) of 64 Gaussians.
Then, they use PCA to reduce the size of the vector and the machine learning method is structured-output multi-class SVM. They use their method on their new dataset named ETHZ Food-101 (56\% accuracy) and MIT-indoor (58\% of accuracy on the full dataset) and compare it against several previous implementations.

ETHZ Food-101 \footnote{Dataset can be found at  \url{https://www.vision.ee.ethz.ch/datasets_extra/food-101/}} is composed of 101 categories, 1000 images per category (250 pictures manually reviewed, used for the test set and 750 with noises for the training test). Pictures were extracted from the website \href{http://www.foodspotting.com/}{foodspotting.com}. The top 101 most popular dishes from this social sharing food images defined the categories.

% 8.3

\cite{Chen2012} present a method to automatically identify food and estimate the quantity. It is used on an in-house dataset composed of 50 categories, mainly Chinese fishes, with 100 pictures per class. For recognition, the authors use:
\begin{itemize}
    \item local information with Bag-Of-Words, using SIFT as descriptor and Local binary pattern on a 3-level pyramid
    \item global information: colour histograms and Gabor filters extracted from of each block (image divided into $4 \times 4$ blocks)
\end{itemize}

They train a SVM classifier for each category, then fuse them with the multi-class AdaBoost algorithm. AdaBoost, or \enquote{Adaptive boosting} is a meta-algorithm that combine into a weighted sum multiple classifiers to improve their final performance.
The authors get an overall accuracy of 68.3\%. If we keep the top-3 results, the accuracy is even 90.9\%.


%% DCNN

More recently, people have started to heavily use Convolutional Neural Networks \textit{CNN} with great results.

%% UMPC Food 101
%% 8.1

In \cite{Wang2015}, the authors created a new dataset named UMPC Food-101 ("twin dataset" of ETHZ Food 101) combining text and visual information for recipes. As a proof of concept, they develop a search application for recipe recognition. The user send a query (a food image) and as a result, the three best recipes (categories) are displayed.

UMPC Food-101 \footnote{Dataset can be found at \url{http://visiir.lip6.fr/}} is a \enquote{twin-dataset} of ETHZ Food 101 as it is composed of the same 101 categories, with 1000 images per category. Yet, the pictures have been crawled from Google image, researching for recipes. Thus, most images are associated with a text.

For the image recognition model, they use textual, visual or a mix of both features:
\begin{itemize}
    \item visual feature (all feeding a SVM):
    \begin{itemize}
        \item Bag-of-Words using a dense SIFT and a codebook of size 1024 on a 3-level spatial pyramid. They obtain an average accuracy of 23.96\%.
        
        \item Use an improved version of the Bag-of-Words named \enquote{BossaNova}. It modifies the pooling system; instead of keeping the closest cluster of a SIFT descriptor, it represents it by keeping distances between the descriptor and all the codebook words. Average accuracy of 28.59\%.
        
        \item Use a deep CNN as a feature descriptor, using the 7th layer of a pre-trained CNN ("OverFeat"). Average accuracy of 33.91\%.
        
        \item Use a very deep CNN as a feature descriptor, using th 19th layers ("vgg-verydeep-19"). Average accuracy of 40.21\%.
    \end{itemize}
    
    \item text-feature: use the term frequency - inverse document frequency \textit{tf-idf} method and get 82.06\% accuracy
    
    \item fusion of textual and visual feature: they obtain at most 85.10\% of accuracy, combining the very deep CNN descriptors and tf-idf.
\end{itemize}

%% UEC FOOD
%% 2.3

In \cite{Kawano2014}, the authors use a pre-trained Deep CNN \textit{DCNN} for feature extraction. The DCNN, called \enquote{OverFeat}, \footnote{Can be found \url{http://cilvr.nyu.edu/doku.php?id=code:start}} was trained on ImageNet and is composed of 19 layers. The authors add more conventional image features to obtain feature vectors composed of: 
\begin{enumerate}
    \item a variant of the Histogram of Oriented Gradients \textit{HOG} called \enquote{Root Hog} that is an element-wise square root of the L1 normalized HOG
    \item mean and variance values of each channel of the RGB representation value of pixels from each of 2*2  block
    \item the last two layers of the DCNN
\end{enumerate}
The three descriptors are then encoded in a fisher vector. Using SVM, the authors obtain 72\% of  accuracy for UEC-FOOD 100.

%% UEC FOOD 256

In \cite{Yanai2015}, the authors use a fine-tuned pre-trained DCNN with the large-scale ImageNet dataset for food recognition. The authors obtain 79\% average accuracy for UEC FOOD 100 and 67\% for UEC FOOD-256.

In \cite{Bolanos2016}, the authors also use a fine-tuned pre-trained Deep Neural Network and obtain 63\% accuracy on UEC FOOD-256. Their neural network is fine-tuned on multiple food datasets (UEC FOOD 256, Food 101 and EgocentricFood).

\section{Food intake estimation}

%% Food Log

FoodLog \footnote{\url{http://www.foodlog.jp}} is a website that enables the user to upload pictures of its daily meals to be archived and processed. The goal of this application is to assist the user to keep notes of their meals and balance the nutritional values coming from different kinds of food.

In \cite{Kitamura2008}, the images containing food items are identified by exploiting features related to the HSV and RGB colour domains, as well as the shape of the plate. A SVM classifier is trained to detect food images. More specifically, the images are divided in 300 blocks and each block is classified as \enquote{non-food} (discarded block) or one of the nutritional categories described in the \enquote{MyPyramid} model \footnote{\url{http://www.mypyramid.gov}}.

MyPyramid \cite{MyPyramid} was designed by the United State Department of Agriculture \textit{USDA} in 2005 and was replaced in 2011 by \enquote{MyPlate} \footnote{\url{http://www.choosemyplate.gov}} \cite{MyPlate}. This dietary model is composed of 5 kinds of food: grains, vegetable, meals and beans, milk and fruit. For each group, a recommended intake per day is associated, Fig. \ref{fig:my_pyramid}. Quantity is categorized by \enquote{servings} \textit{SV}, making it simpler to compute and keep log.

\begin{figure}
    \centering
    \includegraphics[scale=0.7]{img/my_pyramid.jpg}
    \caption[USDA MyPyramid original logo]{USDA MyPyramid original logo. Source \href{https://en.wikipedia.org/wiki/Food_pyramid_(nutrition)}{Wikipedia}.}
    \label{fig:my_pyramid}
\end{figure}

In \cite{Aizawa2013} the Support Vector Machine is replaced by a Bayesian Framework \textit{BF}.
The BF is based on the Gaussian Naive Bayesian (suppose independence between every pair of features and the distribution of each feature is assumed to be Gaussian). The BF takes into account the estimation using colour moments and Bag-Of-Feature of SIFT, the prior distribution and the mealtime category (breakfast, lunch and dinner).

In \cite{Kagaya2014}, the authors use a Convolutional Neural Network \textins{CNN} to detect and classify food from a small subset of image loaded in the FoodLog system. Compared to the other conventional methods (use of a feature descriptor such as Bag-of-Words with a classifier, e.g. SVM) described previously, the CNN showed a significantly higher accuracy.

%% Others

An other method to estimate the food intake is to evaluate the food volume.

In \cite{Chen2012}, the authors presents a method that use the depth information of the picture. Once the food has been classified, the area of the food container (bowl, plate) and the depth value of the contained food is computed to obtain the food volume.
Yet, this technique is still limited as it can only be used for non-transparent food, i.e it can't detect some food item such as water or cooked rice, and force the user to have a depth camera (such as Kinect).

In \cite{Almaghrabi2012a}, the authors presents a novel food recognition system that is able to estimate of the nutrition intake. Moreover, they develop a mobile application to easily take pictures and keep track of the user's diet.
To measure the food intake, authors compare before and after eating pictures and use the thumb as the calibration system (it supposes a one-time calibration to know the size of the thumb of the user).
The process to show the intake is:
\begin{enumerate}
    \item the user takes food pictures
    \item get the contour of each picture
    \item recognition of the food using colour, shape and size features with SVM.
    \item volume calculation, that is computed in two steps:
    \begin{enumerate}
        \item user takes a picture from above. Then, the food shape is divided into known shape (rectangle, circle, triangle ...) to compute the area.
        \item user takes a second picture from the side. This is used to compute the height of the food and calculate the overall volume.
    \end{enumerate}
    The system assumes that the plate is white and round.
    \item use a nutrition database to obtain the average calories
\end{enumerate}
If the user has not eaten everything, the entire must be repeated.
The drawbacks of this method is the user have to take several pictures, with one's thumb each time and it has been tested with a limited set of simple food types.

%
% Zhu
%

In \cite{Zhu2010}, the authors develop a mobile application to keep food records of a user that is taking pictures of one's meal. Their method can detect multiple food items in one picture. They use a colour marker (color chequerboard) as an illumination and size indicator.
As in \cite{Almaghrabi2012a}, images obtained before and after foods are eaten are used to estimate the amount of food consumed.

When the user upload a picture, it is segmented, then classified by a back-end server. The estimation (labelled image with food type and volume) are sent back to the user for confirmation.

For segmentation, the authors use connected component analysis, active contours, and normalized cuts. Then, colour and texture features are extracted to feed a SVM classifier. The authors use:
\begin{itemize}
    \item Gabor filters. Gabor filters describe properties related to the local power spectrum of a signal and have been used for texture analysis
    \item 2-D colour histograms of the a* and b* channels of the CIELaB representation. Values are corrected using the colour marker
\end{itemize}
For the volume estimation, the authors use a 3-D volume reconstruction process. The food area is partitioned and assigned to \enquote{geometric classes}, each with their own sets of parameters.

They evaluate their segmentation and classification methods on a very small dataset composed of 63 images and 19 classes. The authors obtain an average accuracy of 89\%.

In \cite{Zhu2015}, their method is named \enquote{multiple hypotheses segmentation and classification} \textit{MHSC}. It is an iterative algorithm composed of a segmentation, description (extraction of features) and classification steps.

For segmentation, the authors first detect salient region, using Canny edge and colour distribution to reject background. Then, they apply a multi-scale segmentation using normalized cut. Small segmented regions are discarded.

On the selected region, the authors used a mixed of global descriptors (first and second moment of each channel for RGB, YCbCr, L*a*b*, and HSV colour spaces, first and second moment of the entropy in RGB, predominant colour descriptor, entropy and two first moments of the Gradient Orientation Spatial-Dependence Matrix, entropy categorization and fractal dimension estimation and estimation of the fractal dimension of the response of different Gabor filter) with local feature (multi Bag-Of-Words using SIFT for RGB, SURF for RGB, SIFT for each channel of the RGB representation and steerable filters).

Each of the 12 descriptor, global and local, is classified independently and assigned a confidence score. A late fusion function (either maximum confidence score or majority vote) is used  to decide the final class. For classification, the authors use K-NN and SVM.

If the total score is inferior to a certain threshold, the overall process is repeated. The confidence score of the previous step is used to improve the segmentation.

Applied on a dataset composed of 83 labels (79 food classes plus \enquote{utensils}, \enquote{glasses}, \enquote{plates}, and \enquote{plastic cups} classes), each class having at least 30 images, they obtain a top-8 accuracy of 75\%, using K-NN with the maximum confidence score.

%% Food Cam
%% 6.3

In \cite{Matsuda2012a}, the authors propose a food recognition system named \textbf{FoodCam} to identify food items of a picture. The presented process is used on a mobile application, the user taking a picture that is transferred to a sever, processed and results are displayed.

The first step is to detect potential region with multiple object detection algorithms. Then, for these regions, several feeatures are extracted and used to feed SVM with Multiple Kernel Learning \textit{MKL} method. To detect candidate regions, the authors use:
\begin{itemize}
    \item Felzenszwalb’s deformable part model (DPM), based on Histogram of Oriented Gradients (HOG).
    \item a circle detector: the image is converted to a gray-scale, contour are extracted using the Canny Edge Detector and circles detected by the Hough Transform
    \item JSEG region segmentation: segment region based on colour. It only keeps circular regions.
    \item whole image, for picture with one large dish
\end{itemize}
Then, it aggregates all the candidate regions to get the bounding box of each food item.

For each region, it extracts multiple common features:
\begin{itemize}
    \item Bag of Feature of SIFT and C-SIFT (sift with colour invariant characteristics)
    \item Spatial pyramid representation: object regions are divided by hierarchical grids. In this paper, the three level pyramid is used: $1 \times 1$, $2 \times 2$, $3 \times 3$. For each grid, a BoF vector is extracted
    \item Histogram of Oriented Gradient (HOG)
    \item Gabor texture
\end{itemize}

After extraction of the feature vectors from each candidate region, a linear SVM trained by MKL is used ($\chi^2$ kernel). Their methods were evaluated on UEC-FOOD 100. For multiple food item images, they obtain 55.8\% classification rate and 68.9\% for single food item pictures.

%% 

In \cite{Kawano2014a}, the authors develop a mobile real-time food recognition system for calorie and nutrition estimation. Contrary to the previous paper, all the calculation are realised on the user smartphone. The recognition takes less than 1 second thanks to the multi-core architecture of modern smartphones.
The user takes a picture and draws bounding boxes around food items. Then, the system refine the segmentation based on the users' rough demarcation using Grabcut.
For each item, it extracts image features and classify the image among the one hundred food classes using a linear SVM. Then, the top five food candidates are shown and the user can select one of the proposition.
This recognition is updated every one second, the direction arrow as presented in Fig. \ref{fig:food_cam} being displayed to help the user improve the result by changing the camera position and direction. To estimate the most suitable direction, the authors use the Efficient Sub-window Search method, a recent and powerful window search algorithm used in object detection.
The mobile application keep records of all the pictures and their approved classification and labelled with the volume estimation. Food intake is estimated thanks to a slider on the bottom-left of the screen.

\begin{figure}
    \centering
    \includegraphics[scale=0.6]{img/foodcam.jpg}
    \caption[Annotated screenshot of the FoodCam application]{Annotated screenshot of the FoodCam application. Source \cite{Kawano2014a}.}
    \label{fig:food_cam}
\end{figure}

Two different descriptors are used:
\begin{itemize}
    \item bag-of-feature, SURF for detection and description, and colour histogram with the $chi^2$ kernel feature map
    \item HOG and a colour patch descriptor (mean and variance of RGB values on a $2 \times 2$ blocks of pixel) encoded using Fisher Victor, a patch encoding strategy using Gaussian mixture models.
\end{itemize}
The authors evaluate these two methods on UEC-FOOD 100. Taking the top 5 classes, they obtain 79\% classification accuracy for colour patches and 68\% for the other.

%% 11.2

In \cite{Bettadapura2015}, the authors develop an application to recognize food items from an image taken by the user in a restaurant. It uses some contextual data (the geolocalisation) to improve the classification. Indeed, they use geolocalisation to get the menu from internet and query Google Search to get images (extract the top 50 pictures) of 15 dishes from the menu. These images are used as weakly-labelled training pictures to improve the recognition accuracy.

The first step is the segmentation to localize the food and ignore the background through hierarchical segmentation. Then, colour moment invariants, hue histograms, Bag of Words of SIFT, RGB SIFT (SIFT component for each RGB channel), C-SIFT (a color invariant SIFT), Opponent-SIFT (SIFT on colour-opponent channels) are used as feature descriptors. For the 4 SIFT representations: they build a codebooks of 100 000 visual words (using k-means clustering, k = 1000) to build Bag-of-Word histogram.

Then, for the image classification, they adopt the SMO-MKL (Sequential minimal optimization - Multiple kernel learning) multi-class SVM (preceded by $\chi^2$ kernel) methods.

It is applied on these two datasets:
\begin{itemize}
    \item PFID to compare to existing recognition systems. Their method obtain 48.5\% accuracy.
    
    \item in-house dataset consisting of images from 10 restaurants (divided in 5 different types of food: American, Indian, Italian, Mexican and Thai). It is made up of 600 pictures, 300 taken with a smartphone, 300 with Google glasses. The overall average accuracy is 63.33\%, only 15.67\% without localization.
\end{itemize}

General process
- divide the dataset in train and test
- learn and evaluate
- feature description
- choose of the classifier

Feature description

presenting different channel representation
RGB, gray, HSV

Hu moment

Cite + write formula

Bag of words

cite who use it first?

overall process

Sift

Surf

Feature detection
- can use sift or surf
- dense grid (cite why it is better)

Descriptor

clustering
- k nearest neighborhood (can also be used to as a classifier)

Classifier

Tree, random forest

Naive bayesian

SVM + kernel trick

cite the first to use kernel trick
cite why i use $\chi^2$

CNN

SGD classifier


\chapter{Classifier}

k-nearest neighborhood

Naive bayesian

SGD classifier + loss function + regularization term% http://scikit-learn.org/stable/modules/sgd.html#mathematical-formulation

\section{Decision tree and random forest}

Decision tree is a simple learning method that can be used for classification or regression. The implementation used of decision tree is based on the CART (Classification and Regression Tree) algorithm.

A decision tree is recursively partitioning the space in a left $P_{left}$ and right $P_{right}$ partitions such that the samples with the same labels are grouped together, i.e. the generated sets with the smallest impurity.

It continues to split until the impurity can't be reduced or some pre-set stopping rules are met. Alternatively, the data are split as much as possible and then the tree is later pruned.

Since the set of splitting rules used to segment the predictor space can be summarized in a tree, these types of approaches are known as decision tree methods. The figure  \ref{fig:decision_tree_simple_example} illustrate a toy example of decision tree.

\begin{figure}[h]
    \includegraphics[scale=0.5]{img/decision_tree_simple_example}
    \caption[Decision tree of for ten elements belonging to two classes]{Decision tree of depth two for ten elements $(X^1, X^2)$ belonging to the black square and white triangle classes}
    \label{fig:decision_tree_simple_example}
\end{figure}

The most used impurity measure's functions are:
\begin{itemize}
    \item \textbf{Gini}: $$H(X_m) = \sum_k p_{mk} (1 - p_{mk})$$
    \item \textbf{Cross-entropy}: $$ H(X_m) = - \sum_k p_{mk} \log(p_{mk}) $$
\end{itemize}

To avoid overfitting, keep the decision tree as simple as possible.

\textbf{Random forest} or Decision forest is build from a number of decision trees. The prediction of the ensemble is given as the averaged prediction of the individual classifiers. Each tree is trained on a random subsets of the training data.

When building these decision trees, each time a split in a tree is considered, a random sample of
$m$ predictors is chosen as split candidates from the full set of $n$ features. A typical value of $m$ is $m \approx \sqrt{n}$.

\section{Support Vector Machine}

 (binary case) + kernel trick + multi-class (one-versus-one or one-versus-all)

\textbf{Support Vector Machine} \textit{SVM} is a method used for classification and regression.

\subsection{Linear SVM}
\subsubsection{Hard margin}

A support vector machine constructs a hyper-plane or set of hyper-planes in a high or infinite dimensional space. Intuitively, a good separation is achieved by the hyper-plane that has the largest distance to the nearest training data points of any class (so-called functional margin), since in general the larger the margin the lower the generalization error of the classifier.

For a 2 classes (value represented as $-1$ and $1$), the hyperplane must verify:
\begin{equation}\label{eq:svm_1}
    \vec{x_i} \cdot \vec{w} + b \geq + 1 \text{ for } y_i = + 1
\end{equation}
\begin{equation}\label{eq:svm_2}
    \vec{x_i} \cdot \vec{w} + b \leq -1 \text{ for } y_i = - 1
\end{equation}
where $\vec{w}$ is the normal to the hyperplane

Combining equation \ref{eq:svm_1} and \ref{eq:svm_2}, we obtain:
$$ \forall i \in {0, \ldots, n}, ~~ y_i (\vec{x_i} \cdot \vec{w} + b) - 1 \geq 0$$
where $y_i = f(\vec{x_i}) = {-1, 1}$

Gemetrically, the distance between the two hyperplane from \ref{eq:svm_1} and \ref{eq:svm_2} is $\frac{2}{\vec{w}}$ (equal width to each side).

Thus, to obtain the hyperplane with the highest margin, we want to maximize:
$$ \underset {\vec{w}, \vec{b}}{\argmax} \frac{2}{\lVert \vec{w} \rVert^2} $$
which is equivalent to minimize:
$$ \underset {\vec{w}, \vec{b}}{\argmin} \frac{1}{2} \lVert \vec{w} \rVert^2 $$

Thus, we obtain a constrained optimization problem.

\subsubsection{Soft Margin}

For the case of non-separable training sets, we introduce a penality parameter $C$, $C \leq 0$ and obtain:
$$
\underset {\vec{w}, \vec{b} \zeta}{\argmin} \frac{1}{2} \lVert \vec{w} \rVert^2 + C \sum_{i = 1}^{n} \zeta_i
\text{ subject to } y_i (\vec{x_i} \cdot \vec{w} + b) \geq 1 - \zeta_i, \zeta_i \geq 0, \forall i \in [1, \ldots, n]
$$

The decision function for new example is:
$$
f(\vec{x}) = \sign (\sum_{s_i \in \text{ support vectors}} w_i \vec{s_i} \cdot \vec{x} + b)
$$
where the support vectors selected sub-set of the training  examples that define the boundary of the hyperplane separation and hence the classification boundary.

To generalize SVM to the case of multi-class, multiple approaches are possible:
\begin{itemize}
    \item \enquote{one-versus-one}: train a separate classifier for each different pair of labels. This leads to $\frac{N (N - 1)}{2}$ classifiers
    \item \enquote{one-versus-all}: train a single classifier per class, with the samples of that class as positive samples and all other samples as negatives
\end{itemize}

\subsection{Non-linear SVM and kernel trick}

The idea of the kernel trick is to transform the initial space to a higher dimensional space where a hyperplane can separate this data.
Kernel trick: use kernel function to implicitly transform datasets to a higher-dimensional using no extra memory, and with a minimal effect on computation time: realise just a dot product.

To use the linear SVM for non-linear data: project the data in a new feature H space thanks to an application and then reserch for maximum margin hyperplan in H
to make sure that the new problem has a unique solution, 
must satisfy the Mercer's condition or simply it must be a positiv-definit matrix

\begin{itemize}
    \item \textbf{Linear} : $k(x, y) = \langle \vec{x} , \vec{y} \rangle + C = x^T y + C$
    \item \textbf{Polynomial}: $k(x, y) = (\gamma \cdot \langle \vec{x} , \vec{y} \rangle + C)^d = (\gamma \times x^T y + C)^d$
    \item \textbf{Radial Basis Function (RBF)}: $k(x, y) = \exp \left( - \gamma \lVert x - y \rVert ^2 \right)$
    \item \textbf{Chi-Square}: $\displaystyle k(x, y) = 1 - \sum_{i=1}^n \frac{(x_i-y_i)^2}{\frac{1}{2} (x_i+y_i)}$
    
    A modified version presented in \cite{Vedaldi2010} of this kernel is the \textbf{Additive Chi-Square} kernel :
    $\displaystyle k(x, y) = \sum_{i=1}^n \frac{2 (x_i - y_i)}{x_i + y_i} $
\end{itemize}

The adjustable parameters of these kernels are $d$, $\gamma$, $C$ and must be choosen according to the problem.

For food classification, the chi square kernel is the most used kernel as it is often combined with histograms. !!CITE!!

\section{Convolutional neural network}

A \textbf{Convolutional Neural Netwrok} \textit{CNN} is a variant of a Neural Network, mainly used for machine learning on pictures. It is inspired by the neural system composed of different layers (made up of multiple neurons) and communication shemes.

Each neuron receives some inputs, performs a dot product and optionally follows it with a non-linearity function. The whole network still expresses a single differentiable score function (linear or not): from the raw image pixels (the input layer) to class scores (output layer). Hidden layers separates these two layers, as described in \ref{fig:nn_3_layer}.

A CNN (and more generally a NN) is trained by backpropagation, applying gradient descent that will update the weights.

It is a powerful, adaptive and noise resilient pattern recognition. The training phase is rather slow but querying it with an unseen example is fairly fast.

\begin{figure}[h]
    \centering
    \includegraphics[scale=0.4]{img/nn_3_layers.jpeg}
    \caption{A regular 3-layer neural network}
    \label{fig:nn_3_layer}
\end{figure}

\begin{figure}[h]
    \includegraphics[scale=0.35]{img/cnn_simple_example.jpeg}
    \caption[Example of a 16-layer deep convolutional neural network]{Example of a 16-layer deep convolutional neural network. The input layer is a whole picture, the output layer is the probability for each possible class. It used a succession of Convolutional, ReLU, Pooling layer with a final Fully connected one.}
    \label{fig:cnn_simple_example}
\end{figure}

The figure \ref{fig:cnn_simple_example} is a simple CNN based on the VGG-NET structure. It is composed of the 4 most popular layers that can be found in a CNN:
\begin{itemize}
    \item \textbf{Convolutional} : layer fiving the name for this type of neural network. It convolves the input image with a set of learnable filters, each producing one feature map in the output image, i.e. it computes a dot product on a neighborhhod of pixels:
    
    $$ y_{i, j} = b + \sum_{l=0}^{n - 1} \sum_{m=0}^{n - 1}  w_{l,m} x_{j+l, k+m} $$
    with:
    \begin{itemize}
        \item $x_{i,j}$ the input activation at position $(x, y)$
        \item $w_{l, m}$ the weights of the neuron
        \item $n \times n$ is the size of the layer
        \item $b$ is the bias value
        \item $y_{i, j}$ the output values of the $j$, $k$th neuron
    \end{itemize}
    
    \item \textbf{Activation layer}: element wise operation.
    
    Example of function: the \textbf{Rectified Linear Unit} \textit{ReLU} defines as:
    $$ f(x) = max(0, x)$$
   
    \item \textbf{Pooling} or subsampling layer: down sampling of the input activation size. It reduces the number of values between the input and the output values of this layer to avoid overfitting the data and reduce the computation time of the neural network.
    
    The most common downsampling operation is max, giving rise to \textbf{max pooling}, here shown with a stride of 2 in figure \ref{fig:max_pooling}.
    
    \begin{figure}[h]
        \includegraphics[scale=0.5]{img/max_pooling.jpeg}
        \caption[Illustration of a max pooling layer of stride 2]{Illustration of a max pooling layer of stride 2, i.e. it selects the maximum value from a $2 \times 2$ square}
        \label{fig:max_pooling}
    \end{figure}
    
    \item \textbf{Fully connected}: compute the class scores. As the name implied, this neuron is connected to all activations from the previous values. For classification, it corresponds to a loss function, a common one is the sigmoid:
    $$ \sigma (x) = \frac{1}{1 + \exp(-x)}$$
\end{itemize}

A CNN can also be used as a feature descriptor if we use the output of the last layers.



Presenting UEC FOOD 100 and 256

\chapter{Methodology}
\section{Classification}

\subsection{Histograms and moments}

For each picture:
\begin{enumerate}
    \item extract the sub-image delimited by the bounding box
    \item resize this sub-image to $224 \times 224$ pixels
    \item extract the histogram of local binary pattern on the grayscale image
    \item extract the joint color histogram for the channel $H$ and $s$ of the HSV (hue, saturation and value) representation
    \item extract the first two moments on the R, G, B, H, S and Gray channels
    \item extract the 7 hu-moment
\end{enumerate}

The feature vectors are then normalized to have all features centered around zero (mean equal to 0) and have unit variance (equal to 1).

Then, apply multiple classifiers:
\begin{itemize}
    \item decision tree
    \item random forest (made up of 500 trees)
    \item SVM
\end{itemize}

hyperparameter optimization: using a grid
Try to optimize the accuracy for each classifier
Separate the dataset in 3, 10 \% for validation, 10 cross validation to select the best parameters
Then 10 cross validations to train and test the classifier

Talk in result: show the best amelioration with hyperparemeter (but in general it only improve it by one or two percents)

\subsection{Bag of words}

For each picture:
\begin{enumerate}
    \item extract the sub-image delimited by the bounding box
    \item resize this sub-image to $224 \times 224$ pixels
    \item detection of keypoints: use of a dense grid
    \item descriptors: Root SIFT. Root SIFT is a simple variant of SIFT, presented in \cite{Arandjelovic2012}. When the SIFT descriptors as been computed for each keypoints, we apply an element wise square root of the L1 normalized SIFT vectors
\end{enumerate}

clustering: using the k-means algorithm to obtain a 2500-word codebook.

For each picture:
compute the histogram of occurence counts of visual words

Kernel trick: use of a variant of the $\chi^2$ kernel named additive $\chi$-squared kernel presented in \cite{Vedaldi2010}

Then we apply the SVM classifier.

\subsection{CNN}

A pre-trained CNN used for image recognition on ImageNet Challenge 2014.

\cite{Simonyan2014}

it is available \footnote{\url{https://gist.github.com/ksimonyan/3785162f95cd2d5fee77/}}.

The model is an improved version of the 19-layer model used by the VGG team in the ILSVRC-2014 competition.

\section{Segmentation}

A pre-trained CNN used for saliency detection.

\cite{zhang2015SOD}

it is available \footnote{\url{https://gist.github.com/jimmie33/339fd0a938ed026692267a60b44c0c58}}.

It is the same model as GoogleNet model. It is composed of 19 layers.

\section{Code}

The code is public \footnote{\url{https://github.com/bnogaret/food_log}}.

Using python 3.5.2 and its scientific stack (numpy, scipy, matplotlib)
For the data structure: pandas \cite{McKinney2010}
For the image processing: scikit-image \cite{VanderWalt2014}
For most of the machine learning: package: sklearn \cite{Pedregosa2012}
For the CNN framework: caffe framework \cite{Jia2014a} (using the pyhton layer)
SIFT implementation: opencv 3.1 \cite{Bradski2000}

Documentation is generated from the python file using sphinx.

\chapter{Evaluation}

\subsection{Environment}

All the code has been run on the \enquote{Astral} high performance computer of Cranfield's university. The operating system is SUSE Linux Enterprise Server 11 (64 bits architecture), with a Linux 3 kernel.

The system is separated in login nodes and compute nodes. There are two \enquote{front-end} login nodes and they contain two Intel E5-2660 (Sandy Bridge - 8 cores) CPUs giving 16 CPU cores and have a total of 192 GB of shared memory. The login nodes enable the user to connect to the system and compile one's program. There are 80 compute nodes, each node having two Intel E5-2660 (Sandy Bridge - 8 cores) CPUs. This is giving a total of 1280 available cores. Each compute node have at least accessed to 64 GB shared memory. Nodes are connected with Infiniband\TM low-latency interconnect.

\subsection{Segmentation metrics}

\footnote{Information on the evaluation system can be found at  \url{http://host.robots.ox.ac.uk/pascal/VOC/voc2012/devkit_doc.pdf}}

To measure the precision of the localization / segmentation algorithm, we use the metrics as defined in \cite{pascalVoc2012}.

Detections are considered true or false positives based on the area of overlap with ground truth bounding boxes. To be considered a correct detection, the \textbf{Intersection over Union} $IoU$ between the predicted bounding box $B_p$ and ground truth bounding box $B_{gt}$ must exceed 50\% by the formula:

$$IoU = \frac{area(B_p \cap B_{gt})}{area(B_p \cup B_{gt})}$$

To simplify the calculation, this formula can be rewritten as:

$$IoU = \frac{area(B_p \cap B_{gt})}{area(B_p) + area(B_{gt}) - area(B_p \cap B_{gt})} $$

Using this metric, we can compute the precision $P$, the recall $R$ and the accuracy $A$ given by:

$$ P =  \frac{T_p}{T_p + F_p}$$
$$ R =  \frac{T_p}{T_p + F_n}$$
$$ A = \frac{T_p}{T_p + F_n + F_p} $$

with:
\begin{itemize}
   \item $T_p$ the number of true positives (the bounding boxes correctly localized)
   \item $F_p$ the number of false positives (the predicted bounding boxes incorrectly localized)
   \item $F_n$ the number of false negative (the ground truth bounding boxes not localized)
\end{itemize}

Note that given the convention from \cite{pascalVoc2012}, if more than one predicted bounding box overlaps the same ground truth bounding box, only one will be considered as $T_P$, the rest will be $F_P$s.

\subsection{Classification metrics}

% http://scikit-learn.org/stable/modules/cross_validation.html
cross validation
accuracy
% http://scikit-learn.org/stable/modules/generated/sklearn.metrics.confusion\_matrix.html
confusion matrix

\section{Results}
\subsection{Food segmentation}

For the three metrics:
Accuracy: 0.73 \%
Precision: 0.74 \%
Recall:  0.79 \%

In \cite{Bolanos2016}, the authors use fine-tuned pre-trained Deep Neural Network and obtain around:
Accuracy: 60 \%
Precision: 80 \%
Recall: 70 \%

\subsection{Classification}

For using 10 fold cross validation
without parameters optimization

using LBP (98 bins) + HS (30 * 30 bins) + mean and variance of each RGB channel + Hu-moments
\begin{itemize}
    \item random forest: 21 \% (250 trees, gini)
    \item decision tree: 6 \% (gini)
    \item k-nearest neighborhood: (k=10, distance metric: minkowski, weights of each neighborhood point: uniform): 10 \% and 16 \% with hyperparameter optimization
    \item SGD classifier:  12 \%
    \item Gaussian Naive Bayesian: 4 \%
    \item Linear SVM: 9 \% (no kernel trick)
    \item AdaBoost with decision tree: 4 \% (SAMME.R algorithm)
\end{itemize}

using a 2500-word codebook, root-sift, k-mean, RF (500 trees): 10 \% 

using the CNN + Random forest (500 trees): 49 \%

In \cite{Bolanos2016}, the authors use fine-tuned pre-trained Deep Neural Network and obtain 63 \% accuracyon UEC FOOD-256.

In \cite{Yanai2015}, the authors use fine-tuned pre-trained Deep Neural Network and obtain 67 \% accuracyon UEC FOOD-256.


\subsection{Segmentation followed by classification}

CNN Segmenter + CNN feature descriptor + RF classifier

Result: 0.27 \% (0.73 \% accuracy for segmentation, 0.37 \% for classifier)

Accuracy: 0.73328912
Precision: 0.74412334
Recall:  0.7963661

accuracy: 0.37
precision: 0.54
recall: 0.45
f1-score: 0.41

Top 5 :
french fries 0.93006986503
beef bowl 0.951754344221
hamburger 0.954545415101
rice 0.989278742795
miso soup 0.989988865517

Least 5:
meatloaf 0.0
grilled eggplant 0.0
mozuku 0.0
chicken cutlet 0.0
tanmen 0.00943396137415

134 35 clear soup || miso soup 0.830188600926
124 35 zoni || miso soup 0.745613969683
156 35 oshiruko or red bean soup || miso soup 0.71717164473
88 35 Japanese tofu and vegetable chowder || miso soup 0.591549254116
135 35 yudofu || miso soup 0.572727220661
89 35 pork miso soup || miso soup 0.568345282853
82 5 cutlet curry || beef curry 0.54411760705
23 22 beef noodle || ramen noodle 0.503703666392
238 11 kaya toast || toast 0.453488319362
153 86 Caesar salad || green salad 0.444444389575

\cite{Bolanos2016} : accuracy 36.84 \%, 54.44 \% precision, Recall 50.86 \%




% Have to present the dataset
% Feature
% Classification tool (Bag of words, SVM, CNN)
% cnn: present the layers, inspiration ...

\chapter{Future work} \label{sec:conclusion}

One of the possible future area of work is using a more accurate feature descriptor and / or classifier. Compared to the litterature, my food recognition is rather low. Using a pre-trained DCNN for food recognition seems a promising tool.

It could be also interesting to use multiple segmentation method to combine them.

Then, it can be added the estimation part that include an calorie estimation or a simplified version based on MyPyramid or MyPlate and an application to take picture and visualize user's record.


%% Back matter
%
% This is where we include appendices and references
%
\appendix

\chapter{Appendix}

\section{RGB to HSV}

Assuming the RGB values have been normalised to be in $[0, 1]$, we have:

$$M = \max (R, G, B)$$
$$m = \min (R, G, B)$$
$$C = M - m$$

$$
H =
\begin{cases}
0 & \text{if $C = 0$}\\
60 \times \left[ \frac{G - B}{C} \mod 6\right] & \text{if $M = R$} \\
60 \times \left[ \frac{B - R}{C} + 2\right] & \text{if $M = G$} \\
60 \times \left[ \frac{R - G}{C} + 4\right] & \text{if $M = B$} \\
\end{cases}
$$

$$
S =
\begin{cases}
0 & \text{if $M = 0$}\\
\frac{C}{M}& \text{otherwise} \\
\end{cases}
$$

$$V = M$$

\section{HSV to RGB}

$$C = V \times S$$
$$X = C \times (1 - \lvert \frac{H}{60} \mod 2 - 1\rvert )$$
$$
(R', G', B') = 
\begin{cases}
(C, X, 0) & 0 \leq H \leq 60 \\
(X, C, 0) & 60 \leq H \leq 120 \\
(0, C, X) & 120 \leq H \leq 180 \\
(0, X, C) & 180 \leq H \leq 240 \\
(X, 0, C) & 240 \leq H \leq 300 \\
(C, 0, X) & 300 \leq H \leq 360 \\
\end{cases}
$$
$$m = V - C$$
$$ (R, G, B) = (R' + m, G' + m, B' + m)$$

% References - you can use BiBTeX inside the 'references' environment if you want.
% For short reference lists, you might find it just as easy to use the bibitem form.
%\begin{references}
%    \bibitem{lamport94}
%        Leslie Lamport,
%        \emph{\LaTeX: a document preparation system}.
%        Addison Wesley, Massachusetts,
%        2nd edition,
%        1994.
%\end{references}

\printbibliography

\end{document}

