\chapter{Implementation}
\section{Classification}
\subsection{Color histogram}

For each picture:
\begin{enumerate}
    \item extract the sub-image delimited by the bounding box
    \item resize this sub-image to $224 \times 224$ pixels
    \item extract the histogram of local binary pattern
    \item extract the joint color histogram for the channel $H$ and $s$ of the HSV (hue, saturation and value) representation
    \item extract the 7 hu-moment: invariant feature for translation, rotation and scale change (as stated in \cite{Hu1962})
\end{enumerate}

Normalized the data to have all features centered around zero (mean of 0) and have unit variance(variance equal to 1).

Then, apply multiple famous classifiers:
\begin{itemize}
    \item decision tree
    \item random forest
    \item AdaBoost with decision tree
    \item k-nearest neighborhood
    \item SVM
    \item SGD Classifier
\end{itemize}

hyperparameter optimization: using a grid
Try to optimize the accuracy for each classifier
Separate the dataset in 3, 10 \% for validation, 10 cross validation to select the best parameters
Then 10 cross validations to train and test the classifier

Talk in result: show the best amelioration with hyperparemeter (but in general it only improve it by one or two percents)

\subsection{Bag of words}

For each picture:
\begin{enumerate}
    \item extract the sub-image delimited by the bounding box
    \item resize this sub-image to $224 \times 224$ pixels
    \item detection of keypoints: use of a dense grid
    \item descriptors: Root SIFT. Root SIFT is a simple variant of SIFT, presented in \cite{Arandjelovic2012}. When the SIFT descriptors as been computed for each keypoints, we apply an element wise square root of the L1 normalized SIFT vectors
\end{enumerate}

clustering: using the k-means algorithm to obtain a 2500-word codebook.

For each picture:
compute the histogram of occurence counts of visual words

Kernel trick: use of a variant of the $\chi^2$ kernel named additive $\chi$-squared kernel presented in \cite{Vedaldi2010}

Then we apply the SVM classifier.

\subsection{CNN}

A pre-trained CNN used for image recognition on ImageNet Challenge 2014.

\cite{Simonyan2014}

it is available \footnote{\url{https://gist.github.com/ksimonyan/3785162f95cd2d5fee77/}}.

The model is an improved version of the 19-layer model used by the VGG team in the ILSVRC-2014 competition.

\section{Segmentation}

A pre-trained CNN used for saliency detection.

\cite{zhang2015SOD}

it is available \footnote{\url{https://gist.github.com/jimmie33/339fd0a938ed026692267a60b44c0c58}}.

It is the same model as GoogleNet model. It is composed of 19 layers.

\section{Code}

The code is public \footnote{\url{https://github.com/bnogaret/food_log}}.

Using python 3.5.2 and its scientific stack (numpy, scipy, matplotlib)
For the data structure: pandas \cite{McKinney2010}
For the image processing: scikit-image \cite{VanderWalt2014}
For most of the machine learning: package: sklearn \cite{Pedregosa2012}
For the CNN framework: caffe framework \cite{Jia2014a} (using the pyhton layer)
SIFT implementation: opencv 3.1 \cite{Bradski2000}

Documentation is generated from the python file using sphinx.