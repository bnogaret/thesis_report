\section{Color histogram}

For each picture:
\begin{enumerate}
    \item extract the sub-image delimited by the bounding box
    \item resize this sub-image to $224 \times 224$ pixels
    \item extract the histogram of local binary pattern
    \item extract the joint color histogram for the channel $H$ and $s$ of the HSV (hue, saturation and value) representation
    \item extract the 7 hu-moment: invariant feature for translation, rotation and scale change (as stated in \cite{Hu1962})
\end{enumerate}

Normalized the data to have all features centered around zero (mean of 0) and have unit variance(variance equal to 1).

Then, apply multiple famous classifiers:
\begin{itemize}
    \item decision tree
    \item random forest
    \item k-nearest neighborhood
    \item SVM
\end{itemize}

\section{Results}

using 5 fold cross validation

using LBP (40 bins) + HS (10 * 10 bins) + Hu-moments
\begin{itemize}
    \item decision tree: 16\% (500 trees, gini)
    \item random forest: 6.6 \% (gini)
    \item k-nearest neighborhood: (k=10, distance metric: minkowski, weights of each neighborhood point: uniform): 9\%
    \item SGD classifier:  5.6 \%
\end{itemize}

\section{Bag of words}

For each picture:
\begin{enumerate}
    \item extract the sub-image delimited by the bounding box
    \item resize this sub-image to $224 \times 224$ pixels
    \item detection of keypoints: use of a dense grid
    \item descriptors: Root SIFT. Root SIFT is a simple variant of SIFT, presented in \cite{Arandjelovic2012}. When the SIFT descriptors as been computed for each keypoints, we apply an element wise square root of the L1 normalized SIFT vectors
\end{enumerate}

clustering: using the k-means algorithm to obtain a 1000-word codebook.

For each picture:
compute the histogram of occurence counts of visual words

Kernel trick: use of a variant of the $\chi^2$ kernel named additive $\chi$-squared kernel presented in \cite{Vedaldi2010}

Then we apply the SVM classifier.
