\chapter{Dataset}

Why do we use a dataset?
- learning
- some research make them freely available to test

Presenting UEC FOOD 100 and 256

The UEC FOOD-100 dataset can be found in \footnote{Dataset can be found at \url{http://foodcam.mobi/dataset100.html}}. It was created in 2012 and presented in \cite{Matsuda2012a}.

It contains 100 types of food, mainly Japanese food. Each kind is represented by at least 100 pictures.
As presented in figure ..., a photo can contain more than one food items. The dataset contains files to indicate bounding boxes marking the location of a food items.

UEC FOOD-256 can be found in \footnote{Dataset can be found at \url{http://foodcam.mobi/dataset256.html}}. It was presented in \cite{Kawano2015} in 2015. It contains  the 100 types of food from UEC FOOD-100 plus 156 new ones. The newly introduced food kinds are more international food. As for FOOD 100, every food photo has a bounding box indicating the location of the food item.