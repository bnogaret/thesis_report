\chapter{Introduction}

Risk of obesity \cite{Mokdad2003} (uk, world) "Overweight and obesity were significantly associated with diabetes, high blood pressure, high cholesterol, asthma, arthritis, and poor health status". Obesity is strongly associated with several major health risk factors.

Diabetes:
- fast growing (current ... in to ... in) with forecast for ... to be ...
- lead to high mortality
- treatment cost. \cite{Zhang2010}: in 2010: 12 \% of the total whorlwide health expenditure is spent on diabetes and will continue to increase.

Combination of drugs and food intake control have shown great results

Main reason: junk food: easily found, cheap.

One of the best way to fight it: watch over what we eat. Associated lifestyle changes and lose weight. Use as a prevention tool for population at risk

studies such as \cite{Burke2011a} show the benefit of reporting its daily diet to lose weight and improve the quality of its foos intake

Also a way to ... eat disorders

Currently, manually ... self reporting, using paper diaries : hard to do + cost a lot + have some problem (people tend to underestimate) + need a trained patient
tedious, prone to error as the useer tend to under-estimate its intake

At the same time, improvement of the classification methods: example on Image Net results \cite{Russakovsky2015}:
1000 classes, more than 1,2 million images
Every year since 2010
Numerous institutions (university, tech companies) are participated
As described in figure ..., the mean error for each class for classification and localization has been greatly reduced between 2010 and 2014

Recently: proposition automize it. With the widespread use of smartphone, people can easily take pictures of a good quality. People are already taking picture of their food and posting them on website such as Food Gawker, Instagaram, Flickr, Yelp or 

That's why, over the past few years, people ... automated it. Assist patient and their medical personnel
Extends the reach of care in a cost effective ways and counters some of the previous problem (still pb with the elder / people who don't have access to smartphone)

Part of the rise of e-healthcare / m-healtchare \cite{Hillestad2005, Menachemi2011}

food recognition: promising applications of image processing and machine learning. Estimate food intake and people's habit

Overall process:
extract characteristic (possible features are invariant of the liminosity, orientation, scale, ...)segmentate, classify, get calori value or a simplified version (using for example the ... systems), keep log and beng able to visualize it over the year

Feature description: key to achieve good object detection and image categorization

In this thesis: focus on the first two phases

Already have numerous challenges:
large number of food items
variation in appearance and shape
different way to server it
environmental condition
--> lead to a high inter-class variability
challenging task for the human


% show pictures as examples
