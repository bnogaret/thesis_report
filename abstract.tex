\begin{abstract}
    
    Automated food log is a promising exemplar of image analysis which allow users to keep pictures that are processed to keep track automatically of one's food intake. It is a challenging problem due to the high variability of dishes (picture conditions, various types, plating).
    
    With this purpose, the presented master thesis describe a process for simultaneous localisation and recognition. To tackle this problem, several feature descriptors and classifiers were sought to obtain the highest efficiency. From the experiments, the leading method is based on two steps, with first a convolutional neural network pre-trained to detect salient objects is applied on each image to generate bounding boxes for each food area and second, an additional convolutional neural network is used in combination of random forest to recognize the food in each bounding box.
    
    Evaluated on the UEC-FOOD 256 dataset, the method enhances the current best segmentation algorithm with 74\% of top-1 accuracy. Overall, an accuracy of 28 \% were obtained.
    
    \section*{Keywords}
    Food log; Photo; Localisation; Classification; Convolutional neural network
\end{abstract}